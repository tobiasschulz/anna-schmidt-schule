%% LyX 1.5.6 created this file.  For more info, see http://www.lyx.org/.
%% Do not edit unless you really know what you are doing.
\documentclass[ngerman]{scrartcl}
\usepackage[T1]{fontenc}
\usepackage[latin9]{inputenc}
\setlength{\parskip}{\medskipamount}
\setlength{\parindent}{0pt}
\usepackage{babel}

\begin{document}

\part*{Religion}


\section*{14.01.09}

Wiederholung:

Islamische, J�dische, Christliche Credos:

\begin{itemize}
\item Monotheistische Religion
\item Dreifaltiger Gott (Chr.)
\item Mohammed der Prophet Gottes (Isl.)
\item Jesus als Mensch gewordener Gott

\begin{itemize}
\item Passion und Kreuzestod als zentralen Element
\item Heilsgeschichte (Chr./J�d.) (Thema im n�chsten Halbjahr)
\end{itemize}
\item Unterscheidung des Christentums von anderen Religionen
\item Und: Respekt, Toleranz, Zusammenarbeit
\item Theologische Position:

\begin{itemize}
\item Mt 28; 1 Petr. (s.o.)
\item {}``Dignitatis Humanae'', 2. Vatikan. Konzil (1962-65)
\end{itemize}
\end{itemize}
Fasse die wesentlichen Aussagen entlang der Begriffe W�rde, Wahrheit,
Freiheit, Recht und Pflicht zusammen.

(Definition Lehramt: Papst und Bisch�fe, die zusammen entscheiden,
was die authentische christliche Lehre ist).

\begin{itemize}
\item Zusammenfassung:

\begin{itemize}
\item Jeder Mensch hat das Recht auf religi�se Freiheit, das in der Gesellschaft
so anerkannt werden sollte, dass es zum b�gerlichen Recht wird. Durch
ihre W�rde haben die Menschen selbst die Pflicht, ihre pers�nliche
Wahrheit der Religion gegen�ber zu suchen. Diese Wahrheit muss der
Mensch auf eine der W�rde des Menschen angemessene Art selbst erkennen.
Das kann durch die freie Forschung, mit Hilfe des Lehramtes oder durch
Kommunikation mit anderen Menschen geschehen. Der Mensch ist dann
verpflichtet, sein ganzes Leben nach diesen Wahrheiten auszurichten.
\end{itemize}
\item Zusammenfassung von Herrn Riedel:

\begin{itemize}
\item Der Mensch besitzt eine unverlierbare W�rde. -> Freiheit, sein Leben
eigenverantwortlich zu gestalten. -> insbesondere auch religi�se Lebensgestaltung,
d.h. die Wahrheit im Bereich der Religion zu suchen
\item d.h. positive und negative Religionsfreiheit in der Gesellschaft!
\item aber auch: moralische Pflicht zur Wahrheits-/Gott-Suche
\item und dazu geeignete Mittel und Wege w�hlen...
\end{itemize}
F�r die Kiche ist die katholische/christliche Religion die einzig
wahre Religion.\end{itemize}

\end{document}
