%% LyX 1.6.2 created this file.  For more info, see http://www.lyx.org/.
%% Do not edit unless you really know what you are doing.
\documentclass[12pt,english]{article}
\usepackage{mathptmx}
\renewcommand{\familydefault}{\rmdefault}
\usepackage[T1]{fontenc}
\usepackage[latin9]{inputenc}
\usepackage[letterpaper]{geometry}
\geometry{verbose,tmargin=1cm,bmargin=2cm,lmargin=1cm,rmargin=1cm,headheight=1cm,headsep=1cm,footskip=1cm}
\setlength{\parskip}{\medskipamount}
\setlength{\parindent}{0pt}
\usepackage{textcomp}
\usepackage{relsize}

\makeatletter

%%%%%%%%%%%%%%%%%%%%%%%%%%%%%% LyX specific LaTeX commands.
\DeclareRobustCommand{\lyxmathsym}[1]{\ifmmode\begingroup\def\b@ld{bold}
  \def\rmorbf##1{\ifx\math@version\b@ld\textbf{##1}\else\textrm{##1}\fi}
  \mathchoice{\hbox{\rmorbf{#1}}}{\hbox{\rmorbf{#1}}}
  {\hbox{\smaller[2]\rmorbf{#1}}}{\hbox{\smaller[3]\rmorbf{#1}}}
  \endgroup\else#1\fi}


\makeatother

\usepackage{babel}

\begin{document}

\section*{Wendepunkte}

Betrachtet man die Graphen einer Funktion $f$ und dessen Ableitung
$f'$, so f�llt auf, dass sich dort ein Links-rechts-Wendepunkt befindet,
wo das Maximum der Ableitung zu sehen ist. An der Stelle, an der ein
Rechts-links-Wendepunkt existiert, befindet sich das Minimum. Daher
muss man nach den Extremwerten der ersten Ableitung suchen, wenn man
die Wendepunkte der Originalfunktion sucht. An den Extremwerten ist
die Steigung, also der Funktionswert der zweiten Ableitung, 0.


\subsection*{Das notwendige Kriterium f�r Wendepunkte}

Die Stelle $x_{w}$ ist ein Wendepunkt der Funktion $f$, wenn $f$
an der Stelle $x_{w}$ zweimal differenzierbar ist und $f''(x_{w})=0$
gilt.


\subsection*{Das hinreichende Kriteritum f�r Wendepunkte (I)}

Die Funktion $f$ sei in einer Umgebung von $x_{w}$ dreimal differenzierbar.

Gilt $f''(x_{w})=0$ und $f'''(x_{w})\neq0$, so liegt an der Stelle
$x_{w}$ein Wendepunkt von f.

Genauer:
\begin{enumerate}
\item $f'''(x_{w})<0$ $\Longrightarrow$ Links-rechts-Wendepunkt
\item $f'''(x_{w})>0$ $\Longrightarrow$ Rechts-links-Wendepunkt
\end{enumerate}

\subsection*{Das hinreichende Kriteritum f�r Wendepunkte (II)}

Das erste hinreichende Kriterium versagt seinen Dienst, wenn $f''(x_{w})=0$
und auch $f'''(x_{w})=0$ gilt.

Die Funktion $f$ sei in einer Umgebung von $x_{w}$zweimal differenzierbar
und es sei $f''(x_{w})=0$.

Wenn dann die zweite Ableitung $f''$an der Stelle $x_{w}$ einen
Vorzeichenwechsel hat, so liegt dort eine Wendestelle von f.

Genauer:
\begin{enumerate}
\item Vorzeichenwechsel von $+$nach $-$$\Longrightarrow$Links-rechts-Wendepunkt
\item Vorzeichenwechsel von $-$ nach $+$$\Longrightarrow$Rechts-links-Wendepunkt
\end{enumerate}

\subsection*{Beispiel: S. 182 Nr. 5 a)}


\subsubsection*{1. Ableitungen bestimmen}

$f(x)=\frac{1}{8}x\lyxmathsym{\textthreesuperior}-\frac{3}{8}x\lyxmathsym{\texttwosuperior}$

$f'(x)=\frac{3}{8}x^{2}-\frac{3}{4}x$

$f''(x)=\frac{3}{4}x-\frac{3}{4}$


\subsubsection*{2. Notwendige Bedingung}

$f''(x)=0$

$0=\frac{3}{4}x-\frac{3}{4}$

$x=1$


\subsubsection*{3. Hinreichende Bedingung}

$f'''(x)=\frac{3}{4}\Longrightarrow$Rechts-links-Wendepunkt
\end{document}
