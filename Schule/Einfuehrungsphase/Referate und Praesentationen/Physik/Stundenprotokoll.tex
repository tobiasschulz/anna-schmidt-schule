%% LyX 1.6.2 created this file.  For more info, see http://www.lyx.org/.
%% Do not edit unless you really know what you are doing.
\documentclass[english]{article}
\usepackage[T1]{fontenc}
\usepackage[latin9]{inputenc}
\usepackage[letterpaper]{geometry}
\geometry{verbose,tmargin=1cm,bmargin=1cm,lmargin=1cm,rmargin=1cm,headheight=1cm,headsep=1cm,footskip=1cm}
\setlength{\parskip}{\medskipamount}
\setlength{\parindent}{0pt}
\usepackage{textcomp}
\usepackage{relsize}
\usepackage{graphicx}

\makeatletter

%%%%%%%%%%%%%%%%%%%%%%%%%%%%%% LyX specific LaTeX commands.
\DeclareRobustCommand{\lyxmathsym}[1]{\ifmmode\begingroup\def\b@ld{bold}
  \def\rmorbf##1{\ifx\math@version\b@ld\textbf{##1}\else\textrm{##1}\fi}
  \mathchoice{\hbox{\rmorbf{#1}}}{\hbox{\rmorbf{#1}}}
  {\hbox{\smaller[2]\rmorbf{#1}}}{\hbox{\smaller[3]\rmorbf{#1}}}
  \endgroup\else#1\fi}


\makeatother

\usepackage{babel}

\begin{document}

\subsubsection*{Physik Stundenprotokoll - Tobias Schulz - 06.11.08}


\section*{Seite 31 Nr. 5)}

Wie gro� ist im Beispiel 5c) die Masse der Beifahrerin, wenn sie den
Anfahrweg in 4s von 24m auf 10m verk�rzt?

Gegeben:

$m_{1}=200kg$

$a_{1}=3\frac{m}{s^{2}}$

$a_{Grenze}=2,5\frac{m}{s^{2}}\leftarrow Grenze$

$F_{derFahrer}=600N$

$m_{derFahrer+Fahrzeug}=200kg$

Rechnung:

$m_{maximalesGewicht}=\frac{600kg\frac{m}{s^{2}}}{2,5\frac{m}{s^{2}}}$

Verk�rzung des Wegs auf 10m:

$s=10m=0.5*a*t^{2}$

$t=4s$

$\Longrightarrow s=10m=\frac{1}{2}*1,25\frac{m}{s^{2}}*4^{2}s^{2}$

$600N=(200kg+m_{Beifahrerin})*1,25\frac{m}{s^{2}}$

$600N=250N+m_{Beifahrerin}*1,25\frac{m}{s^{2}}$

$480kg=200kg+m_{Beifahrerin}$

$\Longrightarrow m_{Beifahrerin}=480kg-200kg=280kg$


\section*{Seite 31 Nr. 7)}

Gegeben:

$m_{2}=1kg$

$t=2s$

$s=30cm=0,3m$

Gesucht:

$m_{1}=?$

Rechnung:

$s=0,3m=\frac{1}{2}*2^{2}s^{2}*a=2s^{2}a$

$\Longrightarrow0,3m=2s^{2}a$

$\Longrightarrow0,15\frac{m}{s^{2}}=a$

$F_{links}=10\frac{m}{s^{2}}*1kg=10N$

$F_{rechts}=?$

$G=0.15\frac{m}{s^{2}}*(1kg-m_{rechts})$

$F_{rechts}=F_{links}+G=10N+0,15\frac{m}{s^{2}}*m_{rechts}=10\frac{m}{s^{2}}*m_{rechts}|-0,15\frac{m}{s^{2}}*m_{rechts}$

$10N=m_{rechts}*(10\frac{m}{s^{2}}-0,15\frac{m}{s^{2}})$

$\longrightarrow m_{rechts}=\frac{10N}{10\frac{m}{s^{2}}-0,15\frac{m}{s^{2}}}=1.015228426395939kg$


\section*{Schiefe Ebene}

\includegraphics{\string"schiebe ebene\string".eps}

$\alpha=\delta=\zeta$(=zeta)

$\beta=\epsilon$(=epsilon)

Komponentenzerlegung:

$\overrightarrow{G}\,^{2}=\overrightarrow{F}\,_{H}^{2}+\overrightarrow{F}\,_{N}^{2}+2*\overrightarrow{F}\,_{H}^{}*\overrightarrow{F}\,_{N}^{}*cos\beta$

$|\overrightarrow{F}\,_{H}^{}|=\sin\beta*\overrightarrow{|G}|$

$|\overrightarrow{F}\,_{N}^{}|=\cos\beta*\overrightarrow{|G}|$


\section*{Seite 41 Nr. 4)}

Gegeben:

$m=1000kg$

$G=10000N$

$\alpha=15\lyxmathsym{\textdegree}$

$a=1\frac{m}{s^{2}}$

Gesucht:

$|\overrightarrow{F}\,_{H}^{}|=?$

Rechung:

$|\overrightarrow{F}\,_{H}^{}|=\sin(15\lyxmathsym{\textdegree})*10000N=2588N$

$\overrightarrow{F}\,_{2}^{}=1000kg*1\frac{m}{s^{2}}=1000N$

$\Longrightarrow\overrightarrow{F}\,_{Ziehen}^{}=\overrightarrow{F}\,_{H}^{}+\overrightarrow{F}\,_{2}^{}=3588N$
\end{document}
