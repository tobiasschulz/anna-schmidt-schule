%% LyX 1.6.2 created this file.  For more info, see http://www.lyx.org/.
%% Do not edit unless you really know what you are doing.
\documentclass[10pt,english]{article}
\usepackage{mathptmx}
\renewcommand{\familydefault}{\rmdefault}
\usepackage[T1]{fontenc}
\usepackage[latin9]{inputenc}
\usepackage[letterpaper]{geometry}
\geometry{verbose,tmargin=1cm,bmargin=2cm,lmargin=1cm,rmargin=1cm,headheight=1cm,headsep=1cm,footskip=1cm}
\setlength{\parskip}{\smallskipamount}
\setlength{\parindent}{0pt}

\usepackage{babel}

\begin{document}
�berthema: \hfill{}Musik und Kunst\hfill{}\hfill{}\hfill{}\hfill{}\hfill{}\hfill{}\hfill{}\hfill{}\hfill{}\hfill{}\hfill{}\hfill{}\hfill{}\hfill{}\hfill{}\hfill{}\hfill{}\hfill{}\hfill{}\hfill{}\hfill{}\hfill{}

Referent: \hfill{}Tobais Schulz\hfill{}\hfill{}\hfill{}\hfill{}\hfill{}\hfill{}\hfill{}\hfill{}\hfill{}\hfill{}\hfill{}\hfill{}\hfill{}\hfill{}\hfill{}

Datum: \hfill{}24.03.09\hfill{}\hfill{}\hfill{}\hfill{}\hfill{}\hfill{}\hfill{}\hfill{}\hfill{}\hfill{}\hfill{}\hfill{}


\section*{Pablo Picasso - {}``Violine'' (1912/1913)}


\subsection*{Beschreibung:}
\begin{itemize}
\item 58,5 x 21 x 7,5 cm, heute in der Staatsgalerie Stuttgart
\item Die Violine von Picasso besteht aus Schnur, Bleistift, �l und Pappe
\item Mit ihr k�nnte aber nicht musiziert werden, da Picasso die Volumenverh�ltnisse
ver�ndert hat:

\begin{itemize}
\item Statt einem Resonanzkasten sieht man einen offenen Raum und ein querliegendes
Pappst�ck, das mit h�lzern aussehenden Farben bemalt wurde und an
der Seite um einen Schatten erg�nzt wurde, sodass es scheint, dass
das Pappst�ck eine gewisse Dicke besitzt
\item In den offenen Pappkasten, der 5 bis 6 mal so lang wie breit ist und
an der einen Seite eine Spitze in Form eines gleichschenkligen Dreiecks
hat, wurde in der Mitte der H�he und im zweiten F�nftel in Richtung
der Spitze das scheinbare Holzst�ck hineingesteckt
\item Es sieht so aus, als h�tte Picasso eine korrekte Violine zerlegt und
neu zusammengesetzt um bewusst gegen die �bliche Anordnung zu versto�en
\item Durch die verfremdete Darstellung fragt sich der Betrachter, ob mit
diesem Form- und Materialexperiment wirklich eine Violine gemeint
ist.
\end{itemize}
\end{itemize}

\subsection*{Analyse}


\subsubsection*{Was ist dargestellt?}
\begin{itemize}
\item Die {}``Violine'' wirkt ungeordnet und unruhig, ist allerdings auch
sehr symmetrisch (rechte und linke Seite)
\item Schwerpunkt: in der Mitte der mit holz bemalten Farbe finden
\item Kein besonderer Abschnitt des Bildes wird durch einen goldener Schnitt
markiert.
\item Eine �berschneidung kann man bei den Saiten und bei dem in der Mitte
befestigten {}``Holz'' erkennen
\item Die Proportionen wirken unnat�rlich, bei n�herer Betrachtung wirkt
die Raumwirkung verunkl�rt
\item Eine Staffelung existiert nicht, eine Perspektive l�sst sich ebenfalls
nicht zuordnen.
\item Unter der Violine ist ein Schatten sichtbar, wodurch eine gewisse
Dicke der Platte aus Pappe angedeutet werden soll
\item Ein Hell-Dunkel-Kontrast ist zwischen den schwarzen Elementen und
der hellen (Holz-)Pappe sichtbar.
\item Das Licht scheint von oben auf die {}``Violine'' zu treffen
\item Die Violine bewegt sich nicht
\item Darstellungsweise und Wirklichkeitsbegriff kann man nicht genau zuordnen,
da es sich um eine Skulptur handelt.
\end{itemize}

\subsubsection*{Biografisch-psychologischer Kontext}
\begin{itemize}
\item Nach Berichten von Zeitgenossen waren Picassos Wohnungen und H�user
voll gestopft mit Gitarren, bizarren Flaschen, Tapetenst�cken, afrikanischen
Masken und von ihm als bewundernswert angesehenen Gem�lden, beispielsweise
von Matisse, Rousseau und Cezanne.
\item Picasso ist eher als Bildhauer als als Maler bekannt, tats�chlich
aber gab es eine Wechselwirkung zwischen Malerei und Bildhauerei,
da das eine f�r Picasso immer zur Vervollkommnung des jeweils anderen
diente.
\item Statt wie Michelangelo Materialien wie Stein mit Hammer und Mei�el
zu bearbeiten, verwendete Picasso aufgrund seiner spontanen Einf�lle
spontane Materialien, wie Gips, Holz, �ste, K�rbe, Tapete, Pappe und
Schnur.
\item Diese Arbeitsweise ist eher improvisiert, Picasso hatte auch selten
von Anfang an das endg�ltige Kunstwerk im Kopf, sie ergab sich tats�chlich
erst w�hrend des Schaffens.
\end{itemize}

\subsubsection*{Sozialhistorischer Kontext - der Kubismus}
\begin{itemize}
\item Pablo Picasso entwickelte in den jahren 1907/1908 mit Georges Braque
die erste Form des Kubismus (lat. cubus, W�rfel), den {}``analytischen''
(zerlegenden) Kubismus

\begin{itemize}
\item Im analytischen Kubismus gemalte Werke wirken in geometrische Formen
aufgesplittert und es scheint, als w�rde man das Kunstwerk von mehreren
Seiten aus gleichzeitig betrachten
\end{itemize}
$\:$$\longrightarrow$ aufheben der Raumdarstellung, die sich seit
der Renaissance durchgesetzt hatte.

\item Die zweite Form des Kubismus wird {}``synthetischer'' (zusammengesetzer)
Kubismus genannt:

\begin{itemize}
\item bemalte Leinw�nde, geklebte Papiere (franz. {}``papier coll�s'')
oder auch Holzst�cke werden zu einem Werk kombiniert
\item Sand, Glas und S�gesp�ne k�nnen ebenfalls zu eigenst�ndigen Bildelementen
werden
\item Abgesehen von Picasso und Braque war Juan Gris ein weiterer bekannter
Vertreter:

\begin{itemize}
\item Gris sah die Farbe als selbstst�ndiges Konstruktionselement und brach
alle Regeln der naturalistischen Malerei
\item Er trug freie Farben rein intiutiv auf freie Fl�chen auf
\end{itemize}
\end{itemize}
\item Die Entwicklung des Kubismus, der etwa 1908 aufkam und 1925 endete,
wurde durch die 1905 entwickelte Relativit�tstheorie und 1908 durch
den ersten Film beeinflusst.
\end{itemize}

\end{document}
