%% LyX 1.5.6 created this file.  For more info, see http://www.lyx.org/.
%% Do not edit unless you really know what you are doing.
\documentclass[12pt,english]{article}
\usepackage{mathptmx}
\renewcommand{\familydefault}{\rmdefault}
\usepackage[T1]{fontenc}
\usepackage[latin9]{inputenc}
\usepackage{geometry}
\geometry{verbose,letterpaper,tmargin=1cm,bmargin=2cm,lmargin=1cm,rmargin=1cm,headheight=1cm,headsep=1cm,footskip=1cm}
\setlength{\parskip}{\medskipamount}
\setlength{\parindent}{0pt}
\usepackage{babel}

\begin{document}

\part*{Kunst}


\section*{Portfolio (10. Februar - 03. M�rz)}


\subsection*{Beschreibung: {}``Violine'' von Pablo Picasso}

Die Violine von Picasso besteht aus Schnur, Bleistift, �l und Pappe.
Sie ist alllerdings keine Violine, mit der musiziert werden k�nnte,
da Picasso die Volumenverh�ltnisse ver�ndert hat. Statt einem Resonanzkasten
sieht man einen offenen Raum und ein querliegendes Pappst�ck, das
mit h�lzern aussehenden Farben bemalt wurde und an der Seite um einen
Schatten erg�nzt wurde, sodass es scheint, dass das Pappst�ck eine
gewisse Dicke besitzt. Es enth�lt F-L�cher und eine Befestigung f�r
die Saiten. In den offenen Pappkasten, der 5 bis 6 mal so lang wie
breit ist und an der einen Seite eine Spitze in Form eines gleichschenkligen
Dreiecks hat, wurde in der Mitte der H�he und im zweiten F�nftel in
Richtung der Spitze das scheinbare Holzst�ck hineingesteckt.

Es sieht so aus, als h�tte Picasso eine korrekte Violine zerlegt und
neu zusammengesetzt, bewusst gegen die �bliche Anordnung, wie etwas
auszusehen habe, versto�end. Durch die verfremdete Darstellung fragt
sich der Betrachter, ob mit diesem Form- und Materialexperiment wirklich
eine Violine gemeint ist.


\subsection*{Analyse: {}``Violine'' von Pablo Picasso}

Die {}``Violine'' von Pablo Picasso entstand 1912/1913 aus Schnur,
Bleistift, �l und Pappe, ist 58,5 x 21 x 7,5 cm gro� und befindet
sich heute Staatsgalerie Stuttgart. Sie ist der Gattung {}``Stillleben''
zuzuordnen. Bei der ersten Betrachtung hat man den Eindruck, dass
die {}``Violine'' ungeordnet und unruhig wirkt. Sie ist allerdings
auch sehr symmetrisch.

Einen Schwerpunkt kann man auf den zweiten Blick in der Mitte der
mit holz bemalten Farbe finden, daher ist ein Gleichgewicht nur rechts
und links gegeben. Die rechte und linke Seite der Violine sind fast
symmetrisch. Ein goldener Schnitt liegt nicht vor.

Eine �berschneidung kann man bei den Saiten und bei dem in der Mitte
befestigten Pappe-Holz erkennen. Die Proportionen wirken unnat�rlich,
es scheint, dass Picasso versuchte eine Violine zu bauen ohne sie
jemals n�her als in der Oper gesehen zu haben. Eine Staffelung existiert
nicht, eine Perspektive l�sst sich ebenfalls nicht zuordnen, da es
sich um eine Skulptur handelt.

Unter der Violine ist ein Schatten sichtbar, wodurch eine gewisse
Dicke der Platte aus Pappe angedeutet werden soll. Ein Hell-Dunkel-Kontrast
ist zwischen den schwarzen Elementen und der hellen (Holz-)Pappe sichtbar.

Das Licht scheint von oben auf die {}``Violine'' zu treffen. Die
Proportionen sind eindeutig ver�ndert und verfremdet (s. Beschreibung).
Die Violine bewegt sich nicht; Darstellungsweise und Wirklichkeitsbegriff
kann ich nicht genau zuordnen, da es sich um eine Skulptur handelt.
\end{document}
